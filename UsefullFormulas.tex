\documentclass{article}
\usepackage{amsmath}

\begin{document}
\title{Usefull Formulas for Theoretical Physics}
\date{\today}
\maketitle


\section{Euler Integral}
\paragraph{Euler Gamma Function.}
\[
\Gamma(z) = \int_{0}^{\infty} t^{z-1} e^{-t} \, dt
\]
\paragraph{Euler Beta Function.}
\[
B(x, y) = \int_{0}^{1} t^{x-1} (1-t)^{y-1} \, dt
\]
\paragraph{Properties.}
\begin{itemize}
    \item 
    \[
    \Gamma(z+1) = z \Gamma(z)
    \]
    \item 
    \[
    \Gamma(n) = (n-1)!, \quad \Gamma\left(\frac{1}{2}\right) = \sqrt{\pi}
    \]
    \item 
    \[
    B(x, y) = \frac{\Gamma(x)\Gamma(y)}{\Gamma(x+y)}
    \]
    \item 
    \[
    B(x, y) = B(y, x)
    \]
    \item
    \[
    B(x, y) = 2 \int_{0}^{\frac{\pi}{2}} (\sin t)^{2x-1} (\cos t)^{2y-1} \, dt
    \]
    \item
    \[
    B(n, m) = \frac{(n-1)!(m-1)!}{(n+m-1)!}
    \]
    \item 
    \[
    B(x, y) = \frac{1}{x \binom{x+y-1}{x}}
    \]
\end{itemize}

\section{Riemann Zeta Function and Polylogarithm}
\paragraph{Riemann Zeta Function.}
\[
\zeta(s) = \frac{1}{\Gamma(s)} \int_{0}^{\infty} \frac{t^{s-1}}{e^t - 1} \, dt \quad \text{for } \Re(s) > 1
\]
\paragraph{Polylogarithm Function.}
\[
\text{Li}_s(z) = \frac{z}{\Gamma(s)} \int_0^\infty \frac{t^{s-1}}{e^t/z - 1} \, dt \quad \text{for } \Re(s) > 0, \ |z| < 1
\]
\paragraph{Special Values.}
\begin{itemize}
    \item 
    \[
    \zeta(2) = \frac{\pi^2}{6}, \quad \zeta(4) = \frac{\pi^4}{90}
    \]
    \item 
    \[
    \zeta(s) = \sum^\infty_{n = 0} = \frac{1}{n^s}
    \]
    \item 
    \[
    \text{Li}_s(z) = \sum^\infty_{m = 0} \frac{z^m}{m^s}
    \]
    \item 
    \[
    \text{Li}_1(z) = -\ln(1-z)
    \]
    \item 
    \[
    \text{Li}_2(z) = -\int_0^z \frac{\ln(1-t)}{t} \, dt
    \]
    \item 
    \[
    \text{Li}_s(1) = \zeta(s)
    \]
\end{itemize}

\section{Usefull Limit}
\paragraph{Stirling Formula.}
\[
x! \approx x^{x} e^{-x} \quad\quad\quad \text{for } x \longrightarrow \infty 
\]

\newpage

\section{Group Spin(1,3)}
\paragraph{Definition.}
The group Spin(1,3) is the double cover of the Lorentz group SO(1,3), which represents rotations and boosts in 4-dimensional spacetime. The generators of the Spin(1,3) group are given by the six components of the angular momentum operator and the boost operator, as for the Lorentz group:
\[
J_{i} = \frac{1}{2} \epsilon_{ijk} \sigma_{jk}, \quad K_{i} = i \sigma_{0i}
\]
where \( \sigma_{\mu\nu} = \frac{i}{2} [\gamma_\mu, \gamma_\nu] \) and \( \gamma_\mu \) are the gamma matrices. The generators satisfy the following commutation relations:
\[
[J_i, J_j] = i \epsilon_{ijk} J_k, \quad [K_i, K_j] = -i \epsilon_{ijk} J_k, \quad [J_i, K_j] = i \epsilon_{ijk} K_k
\]
Rappresentation of Spin(1,3) has to be with even dimension.

\subsection{Spinor Representation.}
The spinor representation of the Spin(1,3) group is given in the space of Dirac spinors. The rappresentation is provide by 4x4 matrix, the most important are:

wich are summerized in:


Dirac spinors transform under Lorentz transformations as follows:
\[
\psi \rightarrow e^{-\frac{i}{2} \omega_{\mu\nu} \sigma^{\mu\nu}} \psi
\]
where \( \omega_{\mu\nu}  \) are the parameters of the Lorentz transformation.


\paragraph{Properties.}
\begin{itemize}
    \item
    \[
    \{\gamma^\mu, \gamma^\nu\} = 2 \eta^{\mu\nu} I
    \]
    \item 
    \[
    \gamma_5 = \frac{i}{4}\epsilon_{\mu\nu\rho\sigma} = 
    \]
\end{itemize}

\end{document}
