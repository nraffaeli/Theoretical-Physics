\documentclass[a4paper,12pt]{article}

\usepackage{amsmath}
\usepackage{amssymb}
\usepackage{amsthm}
\usepackage{graphicx}
\usepackage[colorlinks=true,allcolors=blue]{hyperref}
\usepackage{appendix}

\title{Notes on Noether's Theorem}
\author{Nicolò Raffaeli}

\newtheorem*{theorem}{Noether's Theorem}

\begin{document}
\maketitle

\begin{theorem}
For every continuous symmetry generated by local actions, there corresponds a conserved current and vice versa.
\end{theorem}

Let's verify this statement, in the contest of Field Theory, remember:
\begin{equation*}
    \mathcal{S} = \int d^4x \, \mathcal{L}(\phi_r, \partial_\mu \phi_r, x^\mu)
\end{equation*}
\paragraph{Hypothesis.} Obviously we suppose that under some continuous symmetry transformation:
\begin{itemize}
    \item \(\mathcal{L}'(\phi_r', \partial_\mu \phi_r', x') = \mathcal{L}(\phi_r, \partial_\mu \phi_r, x)\)
    \item We consider only two type of variation in the field, leading the fact that they're function:
    \begin{itemize}
        \item A \textbf{Form Variation}, evaluating the function with the same argument \(x^\mu\):
        \[
        \delta\phi_r = \phi_r'(x) - \phi_r(x) 
        \]
        \item A \textbf{Total Variation}, wich consider both form variation and variation in the argument:
        \[
        \delta_T\phi_r = \phi_r'(x') - \phi_r(x)
        \]
    \end{itemize}
\end{itemize}

\paragraph{Lemma.} There is a non trivial relation between the two field variation:
\begin{align*}
    \delta_T\phi_r &= \phi_r'(x') - \phi_r(x) \\
    &= \phi_r'(x') - \phi_r(x') + \phi_r(x') - \phi_r(x) \\
    &= \delta\phi_r + \delta x^\nu \partial_\nu \phi_r + \mathcal{O}(\delta x^2) \\
\end{align*}

\paragraph{Thesis.} The continuous symmetry connected to the variation in the Lagrangian \(\delta\phi_r\) generates the following conserve current:
\[
J^\mu = \frac{\partial\mathcal{L}}{\partial(\partial_\mu\phi_r)} \delta\phi_r
\]

\paragraph{Proof.} In order to be satisfied any variation in the Lagrangian has to be zero. Then:
\begin{align*}
    0 &= \delta\mathcal{L} = \frac{\partial \mathcal{L}}{\partial\phi_r} \delta\phi_r + \frac{\partial \mathcal{L}}{\partial(\partial_\mu\phi_r)}\delta(\partial_\mu\phi_r) + \frac{\partial \mathcal{L}}{\partial x^\mu}\delta x^\mu\\
    &= \frac{\partial \mathcal{L}}{\partial\phi_r} \delta\phi_r + \partial_\mu\left(\frac{\partial\mathcal{L}}{\partial(\partial_\mu\phi_r)}\right)\delta\phi_r - \partial_\mu \left( \frac{\partial\mathcal{L}}{\partial(\partial_\mu\phi_r)}\delta\phi_r\right) + \frac{\partial \mathcal{L}}{\partial x^\mu}\delta x^\mu\\
    &= - \partial_\mu \left( \frac{\partial\mathcal{L}}{\partial(\partial_\mu\phi_r)}\delta\phi_r\right) + \frac{\partial \mathcal{L}}{\partial x^\mu}\delta x^\mu\\
    &= - \partial_\mu \left( \frac{\partial\mathcal{L}}{\partial(\partial_\mu\phi_r)}\delta\phi_r - \mathcal{L}\,\delta x^\mu \right) \\
    &= - \partial_\mu J^\mu
\end{align*}
At the third line we have used the Euler-Lagrange equation.

\section*{Poincarè Group}
The general transformation of the Poincarè group is:
\[
x^\mu = \Lambda^\mu_\nu x^\nu + a^\mu
\]
In order to understand the conserved current that this group structure impliais, let's start from the pure translation, and then develop the rotational part.

\subsection*{Translation}
The conserved current came from:
\begin{align*}
    J^\mu &= -\frac{\partial\mathcal{L}}{\partial(\partial_\mu\phi_r)} \delta\phi_r + \mathcal{L}\,\delta x^\mu
\end{align*}
Translation do not involve total variation:
\[
\delta_T\phi_r = 0
\]
It follows from the lemma:
\[
\delta\phi_r = - \delta x^\nu \partial_\nu \phi_r
\]
Then:
\[
J^\mu = \left[ -\frac{\partial\mathcal{L}}{\partial(\partial_\mu\phi_r) } \partial^\nu \phi_r + g^{\mu\nu}\mathcal{L} \right] \delta x_\nu
\]
\[
T^{\mu\nu} = -\frac{\partial\mathcal{L}}{\partial(\partial_\mu\phi_r) } \partial^\nu \phi_r + g^{\mu\nu}\mathcal{L}  
\] 
The stress-energy tensor.

\paragraph{Conserved quantities.} The conserved quantity associated to this current is the four-momentum:
\[
P^\mu = \int d^3x \, \, T^{0\mu}
\]
Indeed:
\begin{align*}
E = P^0 
 &= \int d^3x \, \left[ \frac{\partial\mathcal{L}}{\partial(\partial_0\phi_r)}\partial_0\phi_r - \mathcal{L} \right] \\
 &=  \int d^3x \, \left[ \pi_r\partial_0\phi_r - \mathcal{L} \right] \\
 &= \int d^3x \, \mathcal{H}
\end{align*}
\begin{align*}
    P_i &= \int d^3x \, \left[ \pi_r\partial_0\phi_i - \mathcal{L} \right] \\
\end{align*}
with \(\pi_r\) the canonical momentum of the field.

\subsection*{Lorentz Transformation}
In this case look closer to the infinitesimal transformation:
\[
x^\mu'= x^\mu + \omega^{\mu\nu} x_\nu  
\]
\[
\phi_r'(x') = \phi_r(x) + \frac{1}{2} \omega_{\alpha\beta}S^{\alpha\beta}_{r s}\phi_s(x)
\]
This now as a total variation easy to describe:
\[
\delta_T \phi_r= \frac{1}{2}\omega_{\alpha\beta}S^{\alpha\beta}_{r s}\phi_s
\]
So from the lemma:
\[
\delta\phi_r = \frac{1}{2}\omega_{\alpha\beta}S^{\alpha\beta}_{r s}\phi_s - \delta x^\nu\partial_\nu\phi_r
\]
We can now write down the conserved current:
\[
J^\sigma = \frac{\partial\mathcal{L}}{\partial(\partial_\sigma\phi_r)}\left( \frac{1}{2}\omega_{\alpha\beta}S^{\alpha\beta}_{r s}\phi_s - \delta x^\nu\partial_\nu\phi_r\right) + g^{\sigma\nu}\mathcal{L} \,\delta x_\nu
\]
It comes from the transformation:
\[
\delta x_\nu = x_\nu' - x_\nu = \omega_{\nu\mu} x^\mu
\]
So, using the fact that \(\omega_{\mu\nu}\) is antisymmetric:
\[
J^\sigma = \frac{1}{2} \omega^{\mu\nu}\left[\frac{\partial\mathcal{L}}{\partial(\partial_\sigma \phi_r)}S_{\mu\nu}^{r s} \phi_s - 2 x^\mu \partial_\nu \phi_r + 2 g^{\sigma\nu} x^\mu \mathcal{L}\right]
\]
Making the last two term correctly antisymmetric we can find a better one that show also the dependence on the stress-energy tensor:
\[
J^{\sigma} = \frac{1}{2} \omega^{\mu\nu}\left[\frac{\partial\mathcal{L}}{\partial(\partial_\sigma \phi_r)}S_{\mu\nu}^{r s} \phi_s - x^\nu T^{\sigma\mu} + x^\mu T^{\sigma\nu}\right]
\]
\[
\mathcal{M}^{\sigma\mu\nu} = \frac{\partial\mathcal{L}}{\partial(\partial_\sigma \phi_r)}S^{\mu\nu}_{r s} \phi_s - x^\nu T^{\sigma\mu} + x^\mu T^{\sigma\nu}
\]
Founding the angular-density-momentum tensor.

\paragraph{Conserved quantities.} Let's dig in this tensor: first of all it's antisymmetric in the \(\mu, \nu\) index, so we expect only six conserved quantities:
\[
L_i = \int d^3x \, \mathcal{M}^{00j}   
\]
\[
S_i = \int d^3x \, \epsilon_{ijk}\mathcal{M}^{0jk} 
\]
For \(i,j \) e \(k\) \( = 1, 2, 3\). This notation are meant to remember that, following the fact that \(M^{00j} \propto x^0T^{0j} - x^jT^{00}\), \(L_i\) are the normally the classical angular momentum while the remaining degrees of freedom are intrinsically associated to the field structure.

\section*{Internal Symmetry}
If the symmetry is not shown in the term of variation in the field variable, we have only \textbf{Form Variation} and the variation of the Lagrangian does not drop over the field variable.

\subsection*{U(1)}
An example is when the field is not Hermitian:
\[
\phi_r \quad\quad\quad\quad \phi^{\dagger}_r = \phi_r^*
\]
The internal symmetry for the right presription of the lagrangian is the following:
\[
\phi_r \longrightarrow e^{-iq\alpha}\phi_r
\]
\[
\phi_r \longrightarrow e^{iq\alpha}\phi_r
\]
Where only \(q\) is a physical unconstrained constant while on \(\alpha\) we could implement very interesting architecture: in gauge theory it is a function.

The conserved current is then:
\[
J^\sigma = -iq \left[\frac{\partial\mathcal{L}}{\partial(\partial_\sigma \phi_r)} \phi_r -  \frac{\partial\mathcal{L}}{\partial(\partial_\sigma \phi^{\dagger}_r)} \phi^\dagger_r \right]
\]
\paragraph{Conserved quantity.} Is the classical charge, in fact, the particle number operator is implemented as \(N = \pi\phi\):
\[
Q = J^0 =  -iq \int d^3x \, \left[\pi_r \phi_r -  \pi^{\dagger}_r \phi^\dagger_r \right]
\]
Since there is a minus sign we could recover the particle-antiparticle paradigm of modern physics.

\subsection*{SU(2)}
This symmetry exist for two-component field non hermitian such as:
\[
\Psi_r = \begin{bmatrix}
\phi_r^1 \\
\phi_r^2 
\end{bmatrix}
\quad\quad\quad
\Psi^{\dagger}_r = \begin{bmatrix}
(\phi_r^1)^* & (\phi_r^2)^* 
\end{bmatrix}
\]
With the Lagrangian correct invariant lagrangian for the following trasfromation:
\[
\Psi_r \longrightarrow e^{i\sigma_j(I^j\alpha^j)}\Psi_r
\]
Where \(I_j\) are three unconstricted physical constant, \(\sigma_j\) are the Pauli matrix that form a rappresentation for the non-abelian group SU(2). We explicit the \(\alpha_j\) dependece that making it a function implement a gauge theory.

\paragraph{Conserved quantities.} We have, in analogy whit the first case, but tree conserved charge.

\subsection*{Chiral Symmetry}
The lagrangian associated to massless Dirac spinor is symmetric under the following transformation:
\[
\mathcal{L} = \bar\psi i \gamma_\mu \partial^\mu \psi
\]
\[
\psi \longrightarrow e^{i\lambda\gamma_5}\psi
\]
\[
\bar\psi = \psi^\dagger\gamma_0\longrightarrow ( e^{i\lambda\gamma_5} \psi)^\dagger\gamma_0 = \bar\psi e^{i\lambda\gamma_5}
\]
with the usual Clifford algebra for gamma matrices of spinor space:
\[
\{ \gamma_\mu, \gamma_\nu \} = 2\eta_{\mu\nu}
\]
\[
\gamma_5 = \frac{i}{4} \epsilon^{\mu\nu\rho\sigma}\gamma_\mu \gamma_\nu \gamma_\rho \gamma_\sigma
\]
The conserved current is a chiral current:
\begin{align*}
J_5^\mu &= \left[ \frac{\partial\mathcal{L}}{\partial(\partial_\mu \psi)}\delta\psi + \frac{\partial\mathcal{L}}{\partial(\partial_\mu \bar\psi)}\delta\bar\psi\right] \\
&= \bar\psi\gamma^\mu i \lambda\gamma_5 \psi \\
&= i \lambda \bar\psi\gamma^\mu\gamma_5\psi
\end{align*}

\end{document}
